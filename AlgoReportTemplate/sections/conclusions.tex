Como conclusión se tiene que programación dinámica es muy parecida a fuerza bruta, sin embargo, es mucho mas eficiente en lo que a tiempo respecta, si nos enfocamos en el espacio que se utiliza, es mejor fuerza bruta, pero sabiendo esto ¿cual es el mejor?, la respuesta mas rigurosa es que depende, depende del uso y de la memoria que tiene el dispositivo que ejecutara el problema, pues si un dispositivo no tiene mucha memoria, el hardware limitara a el algoritmo haciéndolo mas deficiente, ademas si se utiliza fuerza bruta lo esperable es que se tenga que esperar un poco mas, pero si se sabe que los strings de entrada no pasan los 10 caracteres, sabremos que fuerza bruta y programación dinámica tardan tiempos muy parecidos.\\
Sabiendo esto podremos concluir que programación dinámica es mejor si los strings son largos, pero si son cortos lo mejor sería fuerza bruta, ya que de esta forma no se perdería espacio, ni tampoco se estaría sacrificando tanto tiempo.\\
Por ultimo también se concluye que no solo es importante como se implemente un programa, sino que como sera utilizado, en este caso vimos que si los strings eran variados los programas tardaban mas, y si no lo eran, resultaban ser mas rápidos

