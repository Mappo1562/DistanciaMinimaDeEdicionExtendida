\epigraph{\textit{Dynamic programming is not about filling in tables. It's about smart recursion!}}{\citeauthor{algorithms_erickson}, \citeyear{algorithms_erickson} \cite{algorithms_erickson}}
Para este enfoque se utilizó una tabla que indica los costos de cada sub-problema, haciendo que el programa realice menos llamadas recursivas, logrando un mejor rendimiento.\\
En esta implementación a palabra y a objetivo se les agrega un carácter al inicio, el cual es '-', esto es para simplificar el acceso a la matriz de sub-problemas, también tendremos la matriz que estará implementada con un vector de 2 dimensiones, esta estará inicializada con valores -1 en todos los puntos menos en el (0,0), en este punto valdrá 0, que es el caso base, la solución al problema trivial de palabra='' y objetivo = ''.\\

\begin{algorithm}[H]
    \SetKwProg{myproc}{Procedure}{}{}
    \SetKwFunction{DistanciaEdicionPD}{DistanciaEdicionPD}
    \SetKwFunction{set}{set} 
    
    \DontPrintSemicolon
    \footnotesize

    \SetKw{KwVar}{String}
    \KwVar{palabra, objetivo}\;

    \SetKw{KwVar}{Matriz}
    \KwVar{tabla}\;


    % Definición del algoritmo principal
    \myproc{\DistanciaEdicionPD{i, j}}{
    \uIf{$tabla_{ij}$ != -1}{
        \Return $tabla_{ij}$\;  
    }

    \uIf{palabra vacia (i == 0)}{
        $tabla_{ij}$ $\leftarrow$ (costo insertar $objetivo_{j}$) + \DistanciaEdicionPD{i, j-1}\;
        \Return $tabla_{ij}$
    }
    \uIf{objetivo vacio (j == 0)}{
        $tabla_{ij}$ $\leftarrow$ (costo eliminar $palabra_{i}$) + \DistanciaEdicionPD{i-1, j}\;
        \Return $tabla_{ij}$\;
    }
    \uIf{palabra[i] == objetivo[j]}{
        $tabla_{ij}$ $\leftarrow$ \DistanciaEdicionPD{i-1, j-1}
        \Return $tabla_{ij}$
    }

    eliminar $\leftarrow$ \DistanciaEdicionPD{i-1, j} + (costo eliminar $palabra_{i}$)\;

    insertar $\leftarrow$ \DistanciaEdicionPD{i, j-1} + (costo insertar $objetivo_{j}$)\;
    sustituir $\leftarrow$ \DistanciaEdicionPD{i-1, j-1} + (costo sustituir $palabra_{i}$ por $objetivo_{j}$)\;
    transponer $\leftarrow$ $\infty$ \;
    \uIf{i y j son > 0 and $palabra_{i}$ == $objetivo_{j-1}$ and $palabra_{i-1}$ == $objetivo_{j}$}{
        transponer $\leftarrow$ \DistanciaEdicionPD{i-2, j-2} + (costo transponer $palabra_{i}$ por $palabra_{i-1}$)\;
    }
    
    $tabla_{ij}$ $\leftarrow$ (minimo entre eliminar, insertar, sustituir, transponer)\;
    \Return $tabla_{ij}$\;
    }


    \myproc{\set{}}{
        ...\\
        palabra $\leftarrow$ $"-"$ $+$ leertxt\\
        objetivo $\leftarrow$ $"-"$ $+$ leertxt \\
        setear tabla con valores -1 \\
        tabla[0][0] $\leftarrow$ 0 \\
        ...
        }

    \caption{Distancia Minima de Edición - Programación Dinamica}
    \label{alg:mi_algoritmo_2}
\end{algorithm}


El algoritmo con el enfoque de programación dinámica, específicamente utilizando el enfoque top down, es muy similar a fuerza bruta, de hecho de igual forma se calculan todos los caminos posibles, pero en este con este enfoque no se volverá a re-calcular, sino que los resultados parciales (resultados del sub-problema) se irán guardando, ahorrando asi llamadas recursivas y mejorando el rendimiento.\\
Este algoritmo divide el problema en varios sub-problemas, dependiendo de la ultima función que se el ejecuto a la palabra, la manera mas fácil de verlo es que se fija el ultimo carácter y se decide hacer una operación, en donde si se agrega, o se sustituye un carácter, siempre se considerara que es el que beneficia al problema, por ende tendremos una fracción del problema resuelto, faltaría el resto, ese seria el primer sub-problema.\\
Para la complejidad temporal tenemos que lo que buscaremos realizar es llenar la matriz con los sub-problemas, esta tendrá un tamaño de $n*m$ donde n es el tamaño de palabra y m es el tamaño de objetivo, en conclusión su complejidad será:
\[
    T(n,m) \in O(n*m)
\]
Por otro lado para la complejidad espacial tendremos que de espacio extra (ignorando espacio utilizado para los costos) tendremos la matriz de sub-problemas, la cual consta de n filas y m columnas, por ende el espacio estará definido por:
\[
    E(n,m) \in  O(n*m)
\]
\\
Ejemplo de ejecución:
considerando los siguientes costos:
\begin{itemize}
    \item $costo\_sub(a,b) = 3$ si $a \neq b$ y $0$ si $a = b$
    \item $costo\_ins(b) = 1$ para cualquier carácter $b$
    \item $costo\_del(a) = 1$ para cualquier carácter $a$
    \item $costo\_trans(a,b) = 1$ para cualquier par de caracteres adyacentes $a, b$\\
\end{itemize} 
y los strings:
\begin{itemize}
    \item palabra = 'abc'
    \item objetivo = 'bad'\\
\end{itemize} 

como el tamaño de palabra y de objetivo es 3, sin embargo le agregaremos el '-' antes, que como ya fue explicado anteriormente es para acceder de manera mas sencilla a la matriz, sabiendo esto los dos strings quedaran de largo 4, por ende comenzaremos llamando a la función con \textbf{i = 3} y \textbf{j = 3}.\\ 
Para este ejemplo el algoritmo se comportara igual que el de fuerza bruta e ira llenando la matriz de a poco, en este problema es visible que la menor distancia es eliminar 'c', agregar 'd', transponer 'ab', con un costo total de 3, el algoritmo generara la siguiente matriz:

\begin{equation}
\begin{matrix}
    0 & 1 & 2 & 3 \\
    1 & 2 & 1 & 2 \\
    2 & 1 & 1 & 2 \\
    3 & 2 & 2 & 3 \\
\end{matrix}
\end{equation}

de la cual podemos apreciar que el algoritmo efectivamente revisa todos los i y los j posibles, y que el resultado será 3 (DistanciaEdicionPD(2,3) + 1), como ya sabemos el resultado, podemos seguir el camino de regreso, como la primera operación fue 'eliminar' tenemos que ir a (2, 3), que guarda el valor 2, esto nos indica que DistanciaEdicionPD(2,3) = 2, luego la siguiente operación fue 'agregar' en consecuencia visitaremos la matriz en (2,2), que es 1, entonces DistanciaEdicionPD(2,2) = 1, por ultimo la operación que viene sera transponer, la cual tiene un costo de 1, que viene de DistanciaEdicionPD(0,0) + 1, pero como i=0 y j=0, el resultado es 0