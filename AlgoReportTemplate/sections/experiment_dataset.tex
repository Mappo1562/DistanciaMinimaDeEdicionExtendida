Los experimentos realizados fueron muy variados, cada uno tiene un enfoque particular, esto debido a que el problema tiene muchas variables que son importantes de analizar, los enfoques utilizados fueron:

\begin{enumerate}[1]
    \item objetivo fijo, este fue "paralelepipedo" de tamaño 14, y la palabra varió con el proposito de que se ejecutara solo una función, es decir eliminar, insertar, sustituir o transponer, las palabras de entrada fueron "paralelepipedoa", "paralelepiped", "paralelepipewo", "paralelepipeod".
    \item objetivo fijo, este fue "paralelepipedo" de tamaño 14, y la palabra se definio aleatoriamente (con el programa de python), de un tamaño $\pm$ 2 con el proposito de analizar como se comporta en casos mas aleatorios.
    \item objetivo fijo, este fue "abcabcabca" de tamaño 10, y ahora se fueron eligiendo palabras aleatorias (con el programa de python), y se comenzo con que estas tuvieran un tamaño de 5 y luego se fue aumentando de 5 en 5 hasta llegar a tamaño 20, esto para ver como se comporta el tiempo de ejecucion en base al largo de la entrada.
    \item este enfoque es muy parecido al anterior, sin embargo se utiliza un string objetivo que es "abcdefghij", esto genera mas variedad en los caracteres que se utilizaran en la palabra, y se tiene el proposito de ver como aumenta el tiempo de ejecucion con strings con caracteres mas variados.
    \item este enfoque busca hacer crecer la palabra y el objetivo al mismo tiempo, y para estos strings se utilizaron los caracteres 'a', 'b', 'c', esto para ver que tanto afecta que los dos strings aumenten.
    \item este enfoque busca hacer hacer lo mismo que el anterior pero con mas varianza en los caracteres de los strings, objetivo va aumentando en orden alfabetico.
\end{enumerate}

Por ultimo, en el \href{https://github.com/Mappo1562/DistanciaMinimaDeEdicionExtendida}{repositorio} hay un archivo exel que contiene todos los datos medidos, la palabra utilizada, el tiempo medido, el resultado de distancia de edicion, el espacio utilizado y los graficos hechos.