\begin{mdframed}
    \textbf{La extensión máxima para esta sección es de 2 páginas.}
\end{mdframed}
Es importante generar varias muestras con características similares para una misma entrada, por ejemplo, variando tamaño del input dentro de lo que les permita la infraestructura utilizada en ests informe, con el fin de capturar una mayor diversidad de casos y obtener un análisis más completo del rendimiento de los algoritmos.

\begin{mdframed}
    Aunque la implementación de los algoritmos debe ser realizada en C++, se recomienda aprovechar otros lenguajes como Python para automatizar la generación de casos de prueba, ya que es más amigable para crear gráficos y realizar análisis de los resultados. Python, con sus bibliotecas como \texttt{matplotlib} o \texttt{pandas}, facilita la visualización de los datos obtenidos de las ejecuciones de los distintos algoritmo bajo diferentes escenarios.
\end{mdframed}    

\begin{mdframed}    
    Debido a la naturaleza de las pruebas en un entorno computacional, los tiempos de ejecución pueden variar significativamente dependiendo de factores externos, como la carga del sistema en el momento de la ejecución. Por lo tanto, para obtener una medida más representativa, siempre es recomendable ejecutar múltiples pruebas con las mismas características de entrada y calcular el promedio de los resultados.
\end{mdframed}