Ahora se especificara mas acerca de las implementaciones de estos programas, estas se encuentran en el siguiente \href{https://github.com/Mappo1562/DistanciaMinimaDeEdicionExtendida}{repositorio}, en este se encuentra la información de todo lo relacionado con el informe.\\
La implementación tiene 4 archivos de gran importancia, estos son:
\begin{itemize}
    \item \texttt{costos.cpp:}\\ En este sub-programa se tendrán las implementaciones de los costos de las funciones.
    \item \texttt{entrada.txt:}\\ Aquí se ingresara en la primera linea la palabra, y en la segunda el objetivo para resolver el problema.
    \item \texttt{FuerzaBruta/distanciaDeEdicion.cpp}\\ En este archivo se guarda la implementación con fuerza bruta del problema, esta posee dos funciones, las cuales son distanciaEdicion(i,j), y set(), la primera resuelve el problema con fuerza bruta revisando todos los casos posibles, la segunda lee los strings del archivo entrada.txt.
    \item \texttt{prg\_dinamica/distanciaDeEdicion.cpp}\\ En este archivo se guarda la implementación con programación dinámica del problema, esta posee dos funciones, las cuales son distanciaEdicion(i,j), y set(), la primera resuelve el problema con programación dinámica revisando todos los casos posibles pero sin re-calcularlos, y la segunda lee los strings del archivo entrada.txt y a estas se les agrega un '-' al inicio, esto es solo para facilitar el acceso a la matriz, no perjudica a la entrada.
\end{itemize}

También se necesitan 4 archivos txt para los costos, estos tendrán un tamaño donde n = 26 (alfabeto inglés, solo con letras minúsculas) estos son; cost\_insert.txt, cost\_delete.txt, cost\_replace.txt, cost\_transpose.txt, estos estarán guardados en la carpeta costos, y cada uno guardara n (insert, delete) o $n*n$ (replace, transpose) números enteros, de igual manera en el repositorio hay un generador de costos aleatorio.

Para correr los programas, solo hay que modificar entrada.txt, ingresando la palabra y el objetivo deseados, luego desde la carpeta \textbf{tarea2-3} compilar con:\\
g++ -o out fuerzabruta/distanciadeedicion.cpp -Wall\\
si se quiere por fuerza bruta, o\\
g++ -o out prg\_dinamica/distanciadeedicion.cpp -Wall\\
si se quiere por programación dinámica, luego para ejecutar tan solo ingresar el comando:\\
.\textbackslash{}out
