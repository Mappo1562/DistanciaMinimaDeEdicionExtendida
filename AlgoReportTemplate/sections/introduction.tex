En este Informe se solucionara el problema del calculo de distancia entre cadenas extendido, para la solución se utilizaran dos enfoques distintos, uno es programación dinámica y el otr fuerza bruta, programación dinámica consiste en probar todos los casos posibles pero ir guardando las soluciones óptimas de los sub-problemas, de esta manera no re-calcula los valores y solo los utiliza, siendo asi mas rápido al momento de ejecutarse, por otro lado tenemos a la fuerza bruta, que también calcula todas las opciones posibles pero sin guardarlas, por ende repetirá trabajo ya realizado, reduciendo el rendimiento, pero ahorrando espacio.

Este problema es muy importante, ya que puede ser utilizado para el reconocer si una palabra esta mal escrita o no, o para auto-correctores, que cuando la distancia sea muy pequeña se corrija, estas son herramientas que se utilizan en el dia a dia, sin embargo se ignora lo que hacen, solo es una caja negra que resuelve problemas, sin embargo este problema soluciona una parte de eso, este informe buscara que se comprenda mejor este algoritmo, como funciona y que estrategia de programación resulta mejor para el.

Este proyecto es muy significativo ya que ayuda a comprender mejor la diferencia entre programación dinámica y fuerza bruta, y también brinda la oportunidad de comparar sus tiempos de ejecución y espacio utilizado, pero resolviendo el mismo problema.




